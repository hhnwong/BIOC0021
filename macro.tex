\\\\Recently, a new approach in managing CRS has been explored. Macrophages were observed to release catecholamines upon activation by IFN-$\gamma$ released by CAR T cell originated from human T cell in a mouse model and this catecholamine production was essential in cytokine release\citep{CRS}. Application of atrial natriuretic peptide (ANP) or metyrosine (MTR) significantly lowered the levels of catecholamines and certain cytokines (human IFN-$\gamma$, TNF and mouse Il-6, CXCL1, CXCL2) when treating mice with high tumour burdens. However, they died prematurely as a result of progressive disease. Treatment of mice with low tumour burdens with ANP and MTR also significantly lowered levels of catecholamines, human TNF, mouse Il-6 and CXCL1 but the decrease in human IFN-$\gamma$ and Il-2 was not as significant. The antitumour effects and expansion of CAR T cell did not seem to be affected by ANP and MTR but cytokines associated to CRS decreased in concentration, showing the effectiveness of the treatment in managing CRS without affecting the efficacy of CAR T cells. 
\\\\As the experiment was based on a mouse model, clinical trials are required to confirm that ANP and MTR have similar effects in human. The potential of CRS management by ANP, MTR and other molecules that can inhibit the production of catecholamines when the patient has low tumour burdens might provide more options in cytokine toxicity management.

They contribute to the progression of cancer by upregulating \textit{VEGF} gene expression and the associated angiogenesis, the formation of new blood vessels, via the PI3K/AKT signalling pathway\citep{MMP2-VEGF, MMP2-VEGF2, MMP9}. 

Dual expression of the CAR and the CCR2b was observed in almost one quarter of the cells and the existence of technical limitations was suspected as the performance of two transduction events resulted in less than 100$\%$ efficiency\citep{CCR2}.